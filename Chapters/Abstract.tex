Pairing-based cryptography over the elliptic curves is a relative new paradigm in public key cryptography(PKC). 
In general, pairing calculation involves certain elliptic curve named pairing-friendly curve defined over finite extension of prime field.
It is typically defined as bilinear map from rational points of two additive groups to a multiplicative group.
Two mathematical tool named as Miller's algorithm and final exponentiation is mostly involved in pairing calculation.
However, most protocols also requires two more operation in pairing groups named scalar multiplication and exponentiation in multiplicative group.
The above mentioned mathematical tools are the major bottleneck for the efficiency of pairing-based protocols.

Since, the inception at the advent of this century pairing-based cryptography brings monumental amount of research. 
The results of this vast amount of research brought some novel cryptographic application which was not possible before pairing-based cryptography. 
However, computation speed of pairing was very slow to consider them as a practical option.
Years of research from the mathematicians, cryptographers and computer scientists improves the efficiency of pairing.

The security of pairing-based cryptography is not only rely on the intractability of elliptic curve discrete logarithm problem (ECDLP) of additive elliptic curve group but also  discrete logarithm problem (DLP) on multiplicative group.
It is known that key size in cryptography based of ECDLP requires fewer bits than cryptography based on DLP.
Therefore, it is a crucial to maintain a balance in parameter sizes for both additive and multiplicative groups in pairing-based cryptography.
In CRYPTO 2016, Kim and Barbulescu showed a more efficient version of number field sieve algorithm to solve DLP. 
This new attack makes all previous parameter settings to update.

This thesis presents several improvement technics for pairing-based cryptography over two ordinary pairing-friendly curves named KSS-16 and KSS-18.
The motivation behind to work on these curves is, they not widely studied in literature compared to other pairing-friendly curves.
After the extNFS algorithm, the security level of widely used pairing-friendly curves were challenged.
The technics can also be applied on the ordinary pairing-friendly curves.
We also present several improvements in extension field arithmetic operation. 
We implement the proposed improvements in for experimental purpose.
All the sources are bundled in an installable library.

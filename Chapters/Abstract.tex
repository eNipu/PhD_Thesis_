Pairing-based cryptography over the elliptic curves is a relatively new paradigm in public key cryptography (PKC). 
It originates many novel cryptographic protocols that were not without pairing. Among these protocols, ID-Based encryption can be interesting for IoT security since it can support a device's ID as a public key.
It can be helpful in the scenario where key-generation is computationally expensive for small devices.
On the other hand, homomorphic encryption can realize strong security and more concrete privacy of patients information while working with encrypted medical data stored in a cloud data-server.

In general, pairing calculation involves a particular elliptic curve named \textit{pairing friendly} curve defined over a finite extension of prime field.
By definition, pairing is a bilinear map from rational points of two additive groups to a multiplicative group.
Two mathematical tools named as Miller's algorithm and final exponentiation is mostly involved in pairing calculation.
However, most protocols also require two more operations in pairing groups named as scalar multiplication and exponentiation in the multiplicative group.
The above mentioned mathematical tools are the major bottleneck for the efficiency of pairing-based protocols.

Since the inception at the advent of this century, pairing-based cryptography brings a remarkable amount of research. 
The results of this vast amount of research brought some novel cryptographic application which was not possible before pairing-based cryptography. 
However, the computation speed of pairing was very slow to consider them as a practical option.
Years of research from the mathematicians, cryptographers and computer scientists improve the efficiency of pairing.

The security of pairing-based cryptography does not only rely on the intractability of elliptic curve discrete logarithm problem (ECDLP) of additive elliptic curve group but also discrete logarithm problem (DLP) on the multiplicative group.
It is known that the "key" size in cryptography based on ECDLP requires fewer bits than cryptography based on DLP.
Therefore, it is crucial to maintaining a balance in parameter sizes for both additive and multiplicative groups in pairing-based cryptography.
In CRYPTO 2016, Kim and Barbulescu showed a more efficient version of the number field sieve algorithm to solve DLP. 
This new attack makes all previous parameter settings to update.

This thesis presents several improvement technics for pairing-based cryptography over two ordinary pairing-friendly curves, i.e., Kachisa-Schaefer-Scott (KSS) KSS-16 and KSS-18.
The motivation behind to work on these curves particularly KSS-16 is it has not been widely studied in the literature compared to other pairing-friendly curves.
Moreover, after the extNFS algorithm, the security level of widely used pairing-friendly curves was in a challenge.

We have proposed improved sparse multiplication in for both curves which reduce the number of finite field operation in Miller's algorithm of Optimal-Ate pairing.
Our optimization of line evaluation for Optimal-Ate pairing in KSS-16 curve is state-of-the-art.
We also proposed efficient scalar multiplication for adapting GLV-based decomposition.
We derived the fundamental relation to applying the GLV decomposition in KSS-16 curve.
We suggested that 6-dimension GLV for KSS-18 and 4-dimension GLV for KSS-16 can achieve optimal calculation cost.
We substantiated our proposal with detailed theoretic explanation and experimental implementations.
We bundled our implementation into an installable shared software library. 
We can apply our proposed technics to other pairing-friendly curves as well.

We are confident that our proposed methods can substantially improve pairing calculation. Therefore, our research contributes to committing high-level security for sophisticated pairing-based protocols for IoT and security and privacy of medical data in the cloud by using pairing-based homomorphic encryption.

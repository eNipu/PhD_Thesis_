\chapter{Fundamentals}
This chapter: temoporary for commont intoductions
\label{Chapter_intros}

\section{ICCE-TW Introduction}
%Since the inception of elliptic curve cryptography (ECC) it has gained wide acceptance mostly due to its smaller key size and greater security. 
%The equation of elliptic curve is defined as
% \begin{equation}\label{eq:eccdef}
%E:y^2 = x^3 + ax + b,
% \end{equation}
% where $4a^3 + 27b^2 \neq 0$. 
In cryptography research, elliptic curve cryptography (ECC) has gained a wide acceptance due to its smaller key size and greater security. 
In ECC, scalar multiplication (SM) is carried out at the encryption and decryption phases. SM is the major operation in ECC. Let us denote a scalar and rational point by  $s$ and $P$, respectively. Then, the SM is denoted by $[s]P$. In real cases $s$ is significantly large number less than the order of rational point group. Since SM needs a complicated calculation over the definition field such as prime field, an efficient algorithm for SM is needed. Recently, ECC defined over extension field $\F{q}{2}$ with a large prime number $p$ such as more than $2000$ bits is used in some ECC based protocols. On the other hand, pairing based cryptography realizes some innovative application protocol. Pairing based cryptography requires pairing friendly curve which is difficult to generate. Barreto-Naehrig (BN) \cite{SAC:BarNae05} curve is one of the well known pairing friendly curve\cite{EPRINT:FreScoTes06} whose parameters are able to be systematically given. BN curve is mostly used due to its efficiency to realize pairing based cryptography. Thus, this thesis proposes an efficient approach for calculating SM on BN curve particularly defined over extension field $\F{q}{2}$, where $q=p^6$ and $p$ is a prime number by using Frobenious Mapping (FM) for the rational points.


\section{WISA - Introduction}


\section{IEICE SDM Introduction}


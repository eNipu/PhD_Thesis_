
In elliptic curve cryptography (ECC), a scalar multiplication for rational point is the most time consuming operation. This paper proposes an efficient calculation for a scalar multiplication by applying Frobenious Mapping. Particularly, this paper deals with  Barreto-Naehrig curve defined over extension field $\F{q}{2}$, where $q=p^6$ and $p$ is a large prime.

%Scalar multiplication of rational points on elliptic curve (EC) is the most time costly operation carried out in elliptic curve cryptography (ECC). In this paper an efficient way to calculate the scalar multiplication is proposed by applying Frobenious Mapping (FM) on the rational points on Barreto - Naehrig (BN) curve. This paper considers the BN curve to be defined over extension field $\F{q}{2}$ for a very large value of $p$.


\section{Preliminaries}
This section briefly discusses the fundamental arithmetic operations required for elliptic curve cryptography defined over prime field $\Fp$ and its extension field $\F{q}{2}$. In addition, this paper focuses on BN curve defined over $\F{q}{2}$, $q=p^6$.

\subsection{BN curve over prime field $\Fp$}
BN curve is a non\- super-singular (\textit{ordinary}) pairing friendly 
elliptic curve of embedding degree 12 \cite{EPRINT:FreScoTes06}. The equation of BN curve defined over $\Fp$ is given by 
\begin{equation}\label{eq:(BN_curve)}
E:y^2=x^3+b, \quad \mbox{($b \in \Fp$)}.
\end{equation}
where $b \neq 0$. Its characteristic $p$, Frobenius trace $t$ and order $r$ are given by using an integer variable $\chi$ as follows:
\begin{eqnarray}
p(\chi) & = & 36\chi^4-36\chi^3+24\chi^2-6\chi+1, \\
r(\chi) & = & 36\chi^4-36\chi^3+18\chi^2-6\chi+1,\label{eq:bn_degree}  \\
t(\chi) & = & 6\chi^2+1.\label{eq:bn_trace} 
\end{eqnarray} 
From \eqref{eq:bn_degree} and \eqref{eq:bn_trace} we find that the bit size of $r$ is two times larger than $t$. Thus, these parameters generally satisfy $t \ll p \approx r$ and the following relation.
\begin{equation}\label{eq:rpt_relation}
r = p+1-t.
\end{equation}

\subsubsection{Point addition}Let $E(\f{p})$ be the set of all rational points on the curve defined over $\f{p}$ and it includes the point at infinity denoted by $\mathcal{O}$.
Let us consider two rational points $P = (x_P, y_P)$, $Q = (x_Q, y_Q)$, and their addition $R = P + Q$, where $\textit{R} = (x_R, y_R)$ and $P, Q, R\in E(\Fp)$. Then, the $x$ and $y$ coordinates of $R$ is calculated as follows.
\begin{subequations}
\begin{equation}\label{eq:point_solpe}
\lambda = 
\begin{cases}
 \frac{y_Q-y_P}{x_Q-x_P} \quad \mbox{($P \neq Q$ and $x_Q \neq x_P$)},\\
  \frac{3x_P^2}{2y_P} \quad  \mbox{($P = Q$ and $y_P\neq 0$)} ,\\
  \phi \quad \mbox{otherwise.}
\end{cases}
\end{equation}

\begin{eqnarray}\label{eq:point_add}
(x_R ,y_R) & = & ((\lambda^2-x_P-x_Q ),\nonumber \\ 
&  &     (x_P-x_R)\lambda-y_P), \mbox{ if $\lambda \neq 0$}.  \\	
(x_R ,y_R) & = & \cal O \quad \mbox{if $\lambda = 0$}.
\end{eqnarray}
\end{subequations}
$\lambda$ is the tangent at the point on EC and $\cal O$ it the additive unity in $E(\f{p})$. When $P=-Q$ then $P+Q=\cal O$ is called elliptic curve addition (ECA). If $P=Q$ then $P+Q=2R$, which is known as elliptic curve doubling (ECD). 


\subsection{Elliptic curve over extension field $\F{q}{2}$}
At first, let us consider arithmetic operations in $\F{q}{2}$, which is the degree $2$ extension field over $\Fq$. In other words extension field $\F{q}{2}$ is the two dimensional vector space over $\Fq$. Let $\left\lbrace v_0, v_1 \right\rbrace $ be a basis of $\F{q}{2}$, an arbitrary element $\textbf{x} \in \F{q}{2}$ is represented as
\begin{equation}\label{eq:vector_withbasis}
\textbf{x} = x_0v_0 +x_1v_1, \ \mbox{ where $x_i \in \Fq$}.
\end{equation} 
When we implicitly know the basis vectors $v_0$ and $v_1$, \eqref{eq:vector_withbasis} is simply expressed as

\begin{equation}\label{eq:vector_reprentation}
\textbf{x} = (x_0,x_1).
\end{equation}

\subsubsection{Addition and subtraction in $\F{q}{2}$}
For vectors, addition, subtraction, and multiplication by a scalar in $\f{q}$ are carried out by coefficient wise operations over $\f{q}$. Let us consider two vectors $\textbf{x}=(x_0,x_1)$ and $\textbf{y}=(y_0,y_1)$. Then,
\begin{eqnarray}
\textbf{x} \pm \textbf{y} & = &  (x_0 \pm y_0, x_1 \pm y_1), \\
k\textbf{x} & = &  (kx_0,  kx_1),  \mbox{ $k \in \Fq$}.
\end{eqnarray}
 
\subsubsection{Vector multiplication in $\F{q}{2}$}
For a vector multiplication, we simply consider a polynomial basis representation. Let  $f(x)$ be an irreducible polynomial of degree 2 over $\Fq$. Particularly, an irreducible binomial is efficient for calculations. In order to obtain an irreducible binomial, Legendre Symbol $\Leg{c}{q}$ is useful. Consider a non-zero element $c \in \Fq$. If $c$ does not have square roots, $f(x) =x^2 - c$ becomes an irreducible binomial over $\Fq$. In order to judge it, Legendre symbol is generally applied. Then, let its zero be $\omega$, $\omega \in \F{q}{2}$, the set $\left\lbrace 1,\omega \right\rbrace $ forms a polynomial basis in $\F{q}{2}$. Using this polynomial basis, the multiplication of two arbitrary vectors is performed as follows:
\begin{eqnarray}
\textbf{xy} & = & (x_0+x_1\omega)(y_0+y_1\omega)\nonumber\\
& = & x_0 y_0 + (x_0 y_1+x_1 y_0)\omega +x_1y_1\omega^2 \nonumber \\ 
& = &(x_0 y_0 + c x_1 y_1)+ (x_0 y_1+x_1 y_0)\omega.
\end{eqnarray}
In this calculation, we have substituted $\omega^2 - c = 0$, since $\omega$ is a zero of the irreducible binomial $f(x)=x^2-c$.

\subsubsection{Vector inversion in $\F{q}{2}$}
For calculating the multiplicative inverse vector of a non-zero vector $\textbf{x}\in \F{q}{2}$, first we calculate the conjugate of $\textbf{x}$ that is given by  Frobenius mapping (FM)
$\pi_q(\textbf{x}) = \textbf{x}^q$. In detail, $\pi_q(\textbf{x})=\textbf{x}^q$ is the conjugate of $\textbf{x}$ to each other. Then the inverse $\textbf{x}^{-1}$ of \textbf{x} is calculated as follows.
\begin{equation}
\textbf{x}^{-1} = n(\textbf{x})^{-1}(\textbf{x}^q), \label{InvCal}
\end{equation}
where  $\textbf{x}$, $\textbf{x}^q$ are the conjugates and $n(\textbf{x}) \in \mFq$ is their
product. FM of $\textbf{x}$, $\pi_q(\textbf{x}) =  (x_0+x_1\omega)^q$ can be easily calculated using an irreducible binomial as follows:
\begin{eqnarray}\label{eq:FM}
(x_0+x_1\omega)^q & = & \sum_{i=0}^{q} {q\choose i} x_0^{(q-i)}(x_1\omega)^i \nonumber\\
%& = & x_0^p + (x_1\omega)^p \nonumber\\
& = & x_0 + x_1\omega^q  \nonumber \\
%\end{eqnarray}
%We can easily calculate $\omega^p$ as follows:
%\begin{eqnarray}\label{eq:omega}
%\omega^p & = & \omega^{p-1} \omega\nonumber\\
& = & x_0+x_1(\omega^2)^{\frac{q-1}{2}}\omega \nonumber \\ 
& = & x_0+x_1(c)^{\frac{q-1}{2}}\omega \nonumber \\
& = & x_0-x_1\omega,
\end{eqnarray}
where we substituted the modular relation $\omega^q = - \omega $. In other words, the conjugate of $\textbf{x}$ is given as $x_0 - x_1\omega$. Therefore, the calculation procedure for $n(\textbf{x}) = \textbf{x}\pi_q(\textbf{x})$ is as follows:
\begin{eqnarray}\label{eq:Inversion}
n(\textbf{x}) & = & (x_0+x_1\omega)(x_0 -x_1\omega)\nonumber\\
& = & x_0^2 - x_1^2\omega^2 \nonumber \\ 
& = & x_0^2 - cx_1^2.
\end{eqnarray}
Since $n(\textbf{x})$ is given without $\omega$, it is found that $n(\textbf{x})$ is a scalar. Finally, the inversion \eqref{InvCal} is efficiently calculated.


\section{Efficient scalar multiplication}
In the context of pairing-based cryptography especially on BN curve, three groups $\g1, \g2$, and $\mathbb{G}_T$ are considered. Among them, $\g1, \g2$ are rational point groups and $\mathbb{G}_T$ is the multiplicative group in the extension field. They have the same order $r$. Let us consider a rational point $Q\in \g2 \subset E(\F{q}{2})$ as $Q(\textbf{x},\textbf{y}) =(x_0+x_1\omega, y_0+y_1\omega)$. In the case of BN curve, it is known that $Q$ satisfies the following relations:
\begin{eqnarray}\label{eq:Q_rel1}
\big[p+1-t\big]Q & = & \cal O \nonumber \\
\big[t-1\big]Q  & = & \big[p\big]Q.
\end{eqnarray}
\begin{eqnarray}\label{eq:Q_rel2}
[\pi_p -p]Q & = &\cal O \nonumber \\
\pi_p(Q) & = & [p]Q.
\end{eqnarray}
Thus, these relations can accelerate a scalar multiplication in $\g2$. From \eqref{eq:Q_rel2} $\pi_p(Q)= [p]Q$. Substituting $[p]Q$ in \eqref{eq:Q_rel1} we find $[t-1]Q = \pi_p(Q)$. 
%The FM of $Q$, $\pi(Q)$ can be easily computed according to \eqref{eq:FM} as 
%\begin{equation}
%\pi(Q)= (x_0 -x_1\omega, y_0-y_1\omega). 
%\end{equation}
Next, let us consider SM $[s]Q$, where $0 \leq s \leq r$. From \eqref{eq:bn_degree} we know $r$ is the order of BN curve  where $[r]Q=\cal O$. Here, the bit size of $s$ is nearly equal to $r$. As previously said, in BN curve $r$ is two times larger than the bit size of $t$. It means that $s$ is two times larger than the bit size of $t-1$. Therefore, let us consider $[t-1]$-adic representation of $s$ as $s = s_0+s_1(t-1)$, where $s$ will be separated into two coefficients $s_0$ and $s_1$ whose size will be nearly equal to or less than the size of $[t-1]$. Then SM  $[s]Q$ is calculated as follows:
\begin{eqnarray}\label{eq:scalar_mul_Q}
[s]Q & =  & [s_0]Q+[s_1(t-1)]Q \nonumber \\
& =  & [s_0]Q+s_1\pi_p(Q).
\end{eqnarray}
Then, applying a multi-scalar multiplication technique, the above calculation will be efficiently carried out.

\section{Conclusion and future work}
%\lipsum[6]
In this paper, we have introduced an acceleration of scalar multiplication on Barreto-Naehrig (BN) curve defined over 2 degree extension field $\F{q}{2}$, $q=p^6$. We have showed that $[t-1]$-adic representation of large scalar number along with Frobenius mapping (FM) on rational points accelerates SM operation significantly, where $t$ is the Frobenius trace of BN curve. As a future work, we would like to evaluate its computational time with a large prime characteristic as a practical situation.


%\begin{thebibliography}{1}

%\bibitem{BN}
%Paulo S. L. M. Barreto and M. Naehrig, ``Pairing-friendly elliptic curves of prime order," Selected Areas in Cryptography, 12th International Workshop, SAC 2005, Kingston, ON, Canada, August 11-12, 2005, Revised Selected Papers, pages 319-331, 2005. Springer LNCS 3897 (2006).
%\bibitem{EPRINT:FreScoTes06}
%D. Freeman, M. Scott, and E. Teske, ``A EPRINT:FreScoTes06 of pairing-friendly elliptic curves," Cryptography ePrint Archive, Report 2006/372 (2006), http://eprint.iacr.org/2006/372
%\bibitem{PAIRING:NASKM08}
%Y. Nogami, M. Akane, Y. Sakemi, H. Katou, and Y. Morikawa, ``Integer Variable chi-Based Ate Pairing," Pairing- Based Cryptography - Pairing 2008, Second International Conference, Egham, UK, September 1-3, 2008. Proceedings, pages 178-191, 2008. Springer LNCS 5209 (2008).


%\end{thebibliography}



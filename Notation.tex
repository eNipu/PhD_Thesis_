\newpage

\textbf{\huge Notation}

\vspace{10mm}
We use the following notations in this thesis.

\vspace{2mm}
\begin{tabular}{c p{28zw}}
\multicolumn{1}{c} {\vspace{2mm}Notation} & \multicolumn{1}{c}{\vspace{2mm}Description}\\
%\vspace{2mm}
$p$ &  $p > 3$ is an odd prime integer in this thesis.\\
%\vspace{2mm}
$x$ mod $p$ & Modulo operation, the least nonnegative residue of $x$ modulo $p$. Thus $0\leq x\ {\rm mod}\ p\leq p-1$.\\
%\vspace{2mm}
$\Fp$ &  The field of integers mod $p$.\\
%\vspace{2mm}
$\mFp$ &  The multiplicative group of the field $\Fp$, in other words, $\mFp=\{x| \ x\in \Fp\ {\rm and}\ x\neq 0\}$.\\
%\vspace{2mm}
 $\Fpm$ &  The extension field over $\Fp$, where $m$ is the extension degree.\\
%\vspace{2mm}
$\mFpm$ &  The multiplicative group of the field $\Fpm$, in other words, $\mFpm=\{x| \ x\in \Fpm\ {\rm and}\ x\neq 0\}$.\\
%\vspace{2mm}
% $k\in_{R}[1,\ n-1]$ & $1\ls k\ls n-1$ is a randomly selected integer.\\
% \vspace{2mm}
% $S_m-S_{m-1}$ & The difference of the sets $S_m$ and $S_{m-1}$, in other words, $x\!\in\! S_m-S_{m-1}$ means $x\!\in\! S_m$ and  $x\!\notin\! S_{m-1}$.\\
% \vspace{2mm}
% $\lfloor \cdot \rfloor$ &  The floor of $\cdot$ is the greatest integer less than or  equal to $\cdot$. For example, $\lfloor 1 \rfloor=1$ and $\lfloor 6.3 \rfloor=6$.\\
% \vspace{2mm}
% $w(\cdot)$ &  $w(\cdot)$ denotes the Hamming weight of $\cdot$. For binary signaling, Hamming weight is the number of  ``1" bits in the binary sequence. \\
% \vspace{2mm}
% $LW(\cdot)$ & $LW(\cdot)$ denotes the expression  $\left\lfloor\log_2(\cdot)\right\rfloor+w(\cdot)-1$  for convenience.\\
% \vspace{2mm}
% $N_{\bar{m}}^{r\bar{m}}(x)$ & $N_{\bar{m}}^{r\bar{m}}(x)\!=\!x^{1+p^{\bar{m}}+p^{2\bar{m}}+\cdots+p^{(r-1)\bar{m}}}$ denotes  the norm of $x \!\in\! GF(p^{r\bar{m}})$ with respect to $GF(p^{\bar{m}})$. $N_{\bar{m}}^{r\bar{m}}(x)$ is always an element of $GF(p^{\bar{m}})$.\\
% \vspace{2mm}
% $C^m(x)$ & Euler's criterion in an extension field $GF(p^m)$ is expressed by $C^m=x^{(p^m-1)/2}$, where $x\in GF(p^m)$.\\
% \vspace{2mm}
% $\phi(x)$ & $\phi(x)=x^p$ is the Frobenius mapping of $x$.\\
% \vspace{2mm}
%  $\phi^{[i]}(x)$ & $\phi^{[i]}(x)=x^{p^i}$ is $i$th iteration of Frobenius mapping of $x$.\\
% \vspace{2mm}
% $\Phi_{\bar{m}}^{m}$ and $\bar{\Phi}_{\bar{m}}^{m}$ & $\Phi_{\bar{m}}^{m}\!=\!\prod_{i=1}^{(\bar{r}-1)/2}{\!\phi^{[(2i-1)\bar{m}]}(x)}$, $\bar{\Phi}_{\bar{m}}^{m}\!=\!\prod_{i=0}^{(\bar{r}-1)/2}{\!\phi^{[(2i)\bar{m}]}}(x)$, where $x\!\in\! GF(p^m)$.\\
%\vspace{2mm}
%$\#A_m$, $\#M_m$, $\#\varphi$ & $A_m$, $M_m$, and $\varphi$ denote additions, multiplications, and Frobenius mappings, respectively, in  $\Fpm$, and  $\#A_m$, $\#M_m$, and $\#\varphi$ denote the respective numbers of these operations. For example, $M_1$ and $M_2$ mean the multiplication in $\Fp$ and $\F{p}{2}$, respectively.  $\#M_1$ and $\#M_2$ mean the numbers of  multiplications in $\Fp$ and $\F{p}{2}$, respectively.\\
%\vspace{2mm}
%$\varphi_i(x)$ & The Frobenius mapping $\varphi_i(x) = x^{p^i}, x\in \Fpm.$\\
$a\mid b$ and $a\nmid b$ & $a\mid b$ means $a$ divides $b$, and $a\nmid b$ means $a$ does not divide $b$.\\
%$x$ \& $y$ & The bit-and operation.\\
%$x \gg y$ & The bit-shift operation.
\end{tabular}




